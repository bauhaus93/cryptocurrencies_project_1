\documentclass[12pt,a4paper]{article}

\usepackage[cm]{fullpage}
\usepackage{amsthm}
\usepackage{amsmath}
\usepackage{amsfonts}
\usepackage{amssymb}
\usepackage{xspace}
\usepackage[english]{babel}
\usepackage{fancyhdr}
\usepackage{titling}
\renewcommand{\thesection}{Exercise \Alph{section}:}

%%%%%%%%%%%%%%%%%%%%%%%%%%%%%%%%%%%%%%%%%%
% This part needs customization from you %
%%%%%%%%%%%%%%%%%%%%%%%%%%%%%%%%%%%%%%%%%%

% please enter exercise sheet number, your name, matriculation, and a
% list of students which you worked together with here
\newcommand{\name}{Jakob Fischer}
\newcommand{\matriculation}{01427178}
\newcommand{\fellows}{None}

%%%%%%%%%%%%%%%%%%%%%%%%%%%%%%%%%%%%%%%%%%
%           End of customization         %
%%%%%%%%%%%%%%%%%%%%%%%%%%%%%%%%%%%%%%%%%%

\author{\name\ (\matriculation)}

\newcommand{\projnumber}{1}
\newcommand{\Title}{Analysing the Blockchain}
\setlength{\headheight}{15.2pt}
\setlength{\headsep}{20pt}
\setlength{\textheight}{680pt}
\pagestyle{fancy}
\fancyhf{}
\fancyhead[L]{Cryptocurrencies - Project \projnumber\ - Analysing the Blockchain}
\fancyhead[C]{}
\fancyhead[R]{\name}
\renewcommand{\headrulewidth}{0.4pt}
\fancyfoot[C]{\thepage}


\begin{document}
\thispagestyle{empty}
\noindent\framebox[\linewidth]{%
 \begin{minipage}{\linewidth}%
 \hspace*{5pt} \textbf{Cryptocurrencies (WS2017/18)} \hfill Prof.~Matteo Maffei \hspace*{5pt}\\

 \begin{center}
  {\bf\Large Project \projnumber~-- \Title}
 \end{center}

 \vspace*{5pt}\hspace*{5pt} \hfill TU Wien \hspace*{5pt}
\end{minipage}%
}
\vspace{0.5cm}
%%%%%%%%%%%%%%%%%%%%%%%%%%%%%%%%%%%%%%%%%%%%%%%

\begin{center}
  Submission by \textbf{\theauthor}\\[1cm]

  During the development of the project solutions, I have discussed
  problems, solutions, and other questions with the following other
  students:\\
  \fellows %please fill the information above
\end{center}

\section{Finding invalid blocks}
For the rules on blocks, I mostly relied on the source mentioned in the assignment description:
\begin{verbatim}
  https://en.bitcoin.it/wiki/Protocol_rules#.22block.22_messages
\end{verbatim}
Several functions were created, to split up bigger queries, making queries reusable and exploration of the data easier:
\begin{itemize}
   \item get\_tx\_throughputs(): Retrieves difference of input source values to output values for each transaction.
   \item get\_block\_throughputs(): Retrieves throughput difference for each block, by grouping data from get\_tx\_throughputs() over block id.
   \item get\_first\_block\_transactions(): Returns transaction with minimal tx id from each block.
   \item get\_block\_coinbases(): Sums over all tx outputs for the first transaction in each block. Does not check, if it's really a coinbase transaction (input source id = -1).
   \item get\_multiple\_used\_outputs(): Groups over input sources (outputs joined with inputs over output id) and returns all rows where more than 1 row was grouped.
   \item get\_first\_multiple\_usages(): Gets the minimal tx id of txs which have an input source with multiple usages (Only this txs are valid).
\end{itemize}
\newline\newline
Some invalid blocks are detected and inserted into invalid\_blocks by multiple queries (eg. the throughput/coinbase queries, legal range queries), so grouping should be done
to avoid getting duplicate block ids.
\newline\newline
Ideas/Execution:
\begin{itemize}


\item Q1: It should be checked, if the first tx of a block (lowest tx id) has correct input source (-1) and sig id (0), as it's the coinbase tx.

\item Q2: Blocks with more than one coinbase transaction should be queried. These have coinbase transactions (input source id -1, sig id 0) with different tx ids.
A valid block must have exactly one coinbase transaction.

\item Q3: The first tx of a block (coinbase tx) cannot have an output of less than 50 BTC, since 50 BTC is the currently used block creation fee.

\item Q4: For each transaction in a block, it should be checked, if the value of the
input sources is greater or equal than the value of the outputs. I called the difference of input to output for a transaction "throughput".
Each transaction should have a throughput greater or equal to zero. If greater than zero, the surplus is the transaction fee.
If throughput is less than zero, the transaction and it's block are invalid.

\item Q5: For each block, the coinbase transaction (which now is now assumed to be the lowest tx id of a block), must be equal to the sum of the block creation fee (50 BTC)
and the surplus of transaction throughputs.\newline
On the one side, this was violated by transactions with negative throughput (if the block creator doesn't apply a negative transaction fee)\newline
On the other side, block creators may have calculated the coinbase false, by not using the correct block creation fee or not adding transaction fees.
This also query also finds some blocks from the prevous queries.

\item Q6/Q7: Double spending occurs, if the same output is used more than once as an input source (in different, but also in the same transactions). Transactions with input sources already
used in a previous (or the current) transaction are invalid (and so are their blocks).

\item Q8: It should be checked, that every output is used only at a later point in time (higher tx id) as input source.
Otherwise it would be possible to spend money which is not (yet) owned or even existing.

\item Q9: Each output must be within the legal money range, which is between 0 and 2100000000000000 satoshis.
\item Q10: The total money transferred in a single transaction must also be in legal range (This query only yielded the block already found with the single money range query).

\item Q11: The public keys of output/input pairs (connected by output id) must be equal. If not, a person other than the owner of the output has used it as
input source in another transaction. However, if a custom script was used (id = -1) on one or two sides, this constraint could not be checked.\newline

\item Q12/Q13: Keys (pk/sig id) cannot be less than -1. I couldn't find any limitations on key range in the project description (except on custom scripts usage and coinbase txs).
However, real bitcoin public keys always have a prefix byte with value 0x2, 0x3 or 0x4 [1].
So when using one's or two's complement to interpret negative numbers on little endian encoding [2] negative pks shouldn't be possible.
\newline
Sources:
\begin{verbatim}
  [1] https://steemit.com/bitcoin/@b1tma0/bitcoin-public-key-formats
  [2] http://learnmeabitcoin.com/glossary/little-endian
\end{verbatim}
\end{itemize}

\section{UTXOs}
\begin{itemize}
\item For finding unspent outputs, each output should be matched to a possible input (which uses the output as input source).
Outputs, which have no connection to an input via output id are not spent (they are not a input source anywhere).
\item Additionally, outputs which are referred as input source in transactions in it's past are considered unspent,
since the output didn't exist at the time of spending.
\end{itemize}
\section{De-anonymization}

Created functions:
\begin{itemize}
  \item get\_richest\_entity(): Returns the entity id and money from the cluster\_money table with the highest amount of money.
  \item get\_richest\_addresses(): Retrieves all addresses of the richtest entity, using the get\_richest\_entity() function.
\end{itemize}
Execution:
\begin{enumerate}
\item To relate addresses by joint control, inputs of the same transaction (tx id) where matched, and their sig ids related. Coinbase transactions should be ignored.
\item For relations using serial control, only transactions with exactly one input and output are considered. Their sig/pk id are related with one another.
\item The results of the clusterAddresses() function are written into a new table called cluster.
\item Next, the owned money of each cluster is written to a table called cluster\_money, for making later queries easier to perform.
\item The get\_richest\_entity function is used to get the highest money amount of a single entity.
\item The get\_richest\_addresses function is used to get the smallest address of the richtest entity.
\item To get the biggest transaction of the richest entity, the maximum value of an transaction output of the richest entity is selected.
This is done with help of the already created get\_richest\_addresses function.Since
none of the the transactions of the richest entity consist of multiple outputs, no additional grouping on tx ids was done.
\end{enumerate}
\end{document}
