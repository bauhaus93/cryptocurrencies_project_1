\documentclass[12pt,a4paper]{article}

\usepackage[cm]{fullpage}
\usepackage{amsthm}
\usepackage{amsmath}
\usepackage{amsfonts}
\usepackage{amssymb}
\usepackage{xspace}
\usepackage[english]{babel}
\usepackage{fancyhdr}
\usepackage{titling}
\renewcommand{\thesection}{Exercise \Alph{section}:}

%%%%%%%%%%%%%%%%%%%%%%%%%%%%%%%%%%%%%%%%%%
% This part needs customization from you %
%%%%%%%%%%%%%%%%%%%%%%%%%%%%%%%%%%%%%%%%%%

% please enter exercise sheet number, your name, matriculation, and a
% list of students which you worked together with here
\newcommand{\name}{Jakob Fischer}
\newcommand{\matriculation}{01427178}
\newcommand{\fellows}{None}

%%%%%%%%%%%%%%%%%%%%%%%%%%%%%%%%%%%%%%%%%%
%           End of customization         %
%%%%%%%%%%%%%%%%%%%%%%%%%%%%%%%%%%%%%%%%%%

\author{\name\ (\matriculation)}

\newcommand{\projnumber}{1}
\newcommand{\Title}{Analysing the Blockchain}
\setlength{\headheight}{15.2pt}
\setlength{\headsep}{20pt}
\setlength{\textheight}{680pt}
\pagestyle{fancy}
\fancyhf{}
\fancyhead[L]{Cryptocurrencies - Project \projnumber\ - Analysing the Blockchain}
\fancyhead[C]{}
\fancyhead[R]{\name}
\renewcommand{\headrulewidth}{0.4pt}
\fancyfoot[C]{\thepage}


\begin{document}
\thispagestyle{empty}
\noindent\framebox[\linewidth]{%
 \begin{minipage}{\linewidth}%
 \hspace*{5pt} \textbf{Cryptocurrencies (WS2017/18)} \hfill Prof.~Matteo Maffei \hspace*{5pt}\\

 \begin{center}
  {\bf\Large Project \projnumber~-- \Title}
 \end{center}

 \vspace*{5pt}\hspace*{5pt} \hfill TU Wien \hspace*{5pt}
\end{minipage}%
}
\vspace{0.5cm}
%%%%%%%%%%%%%%%%%%%%%%%%%%%%%%%%%%%%%%%%%%%%%%%

\begin{center}
  Submission by \textbf{\theauthor}\\[1cm]

  During the development of the project solutions, I have discussed
  problems, solutions, and other questions with the following other
  students:\\
  \fellows %please fill the information above
\end{center}

\section{Finding invalid blocks}

For the rules on blocks, I relied on the source mentioned in the assignment description:\newline \verb https://en.bitcoin.it/wiki/Protocol_rules#.22block.22_messages
\newline\newline
Several functions were created, to split up bigger queries, make queries reusable and exploration of the data easier:
\begin{itemize}
   \item get\_tx\_throughputs(): retrieves difference of input source values to output values for each transaction.
   \item get\_block\_throughputs(): retrieves throughput difference for each block, by grouping data from get\_tx\_throughputs() over block id.
   \item get\_first\_block\_transactions(): returns transaction with minimal tx id from each block.
   \item get\_block\_coinbases(): sums over all tx outputs for the first transaction in each block. Does not check, if it's really a coinbase transaction (input source id = -1).
   \item get\_multiple\_used\_outputs(): groups over input sources (outputs joined with inputs over output id) and returns all rows where more than 1 row was grouped.
   \item get\_first\_multiple\_usages(): gets the minimal tx id of txs which have an input source with multiple usages (Only this txs are valid).
\end{itemize}

\newline\newline
Some invalid blocks are detected and inserted into invalid\_blocks by multiple queries (eg. the throughput/coinbase queries, legal range queries).
\newline\newline
For each transaction in a block, it should be checked, if the value of the
input sources is greater or equal than the value of the outputs. I called the difference of input to output for a transaction "throughput".
Each transaction should have a throughput greater or equal to zero. If greater than zero, the surplus is the transaction fee.\newline
\newline
For each block, the coinbase transaction (input source id = -1), which in it's check is assumed to be the first transaction of a block, must be equal to the sum of the block creation fee (50 BTC)
and the surplus of transaction throughputs.\newline
On the one side, this could be violated by transactions with negative throughput (and the block creator doesn't apply a negative transaction fee)\newline
On the other side, block creators may have calculated the coinbase false, by not using the correct block creation fee or not adding transaction fees. This also includes cases, where the coinbase transaction was
split into two or more separate transactions, which is not allowed.
A separate query for finding blocks with more than one coinbase transaction (transactions with different tx ids, done by grouping over a block and compare min/max tx ids of input sources with id -1) was also created.\newline
\newline
Double spending occurs, if the same output is used more than once as an input source (in different, but also in the same transactions). Transactions with input sources already
used in a previous (or the current) transaction are invalid (and so are their blocks).\newline
Additionally it should be checked, that every output is used only at a later point in time (higher tx id) as input source.
Otherwise it would be possible to spend money which is not (yet) owned.\newline
\newline
The public keys of output/input pairs (connected by output id) must be equal. If not, a person other than the owner of the output has used it as
input source in another transaction. However, if a custom script was used (sig id = -1) this constraint could not be checked.\newline
\newline
Each output must be within the legal money range, which is between 0 and 2100000000000000 satoshis.\newline
Furthermore, the total money transferred in a single transaction must also be in legal range (This query only yielded the block already found with the single money range query)\newline

\section{UTXOs}

For finding unspent outputs, each output should be matched to a possible input (which uses the output as input source).
Outputs, which have no connection to an input via output id are not spent. Additionally, outputs which are referred as source in transactions
in the past are considered unspent, since the transaction is invalid and the output didn't exist at the time of spending.

\section{De-anonymization}

To relate addresses by joint control, inputs of the same transaction (tx id) where matched, and their sig ids related. Coinbase transactions were ignored.
\newline\newline
For relations using serial control, only transactions with exactly one input and output are considered. Their sig/pk id are related with one another.
\newline\newline
The results of the clusterAddresses() function are written into a new table called cluster.
\newline\newline
Next, the owned money of each cluster is written to a table called cluster\_money, for making later queries easier to perform.
To create this table two functions were created, which return the money owned by a specified entity id and the money for each entity id.
\newline\newline
The get\_richest\_entity function returns the entity id and money from the cluster\_money table with the highest amount of money.
\newline\newline
The get\_richest\_entity function is then used to get the highest money amount of a single entity.
\newline\newline
The get\_richest\_addresses function uses the get\_richest\_entity function to first retrieve the id of the richest entity and then match
the entity id with the cluster table to retrieve all addresses of the richtest entity id.
\newline\newline
The result of the get\_richest\_addresses function is then used to get the smallest address of the richtest entity.
\newline\newline
To get the biggest transaction of the richest entity, the maximum value of an transaction output of the richest entity is selected.
This is done with help of the already created get\_richest\_addresses function.Since
none of the the transactions of the richest entity consist of multiple outputs, no additional grouping on tx ids was done.
\end{document}
